%% defense.tex
%% Copyright 2022 Tom M. Ragonneau
%
% This work may be distributed and/or modified under the
% conditions of the LaTeX Project Public License, either version 1.3
% of this license or (at your option) any later version.
% The latest version of this license is in
%   http://www.latex-project.org/lppl.txt
% and version 1.3 or later is part of all distributions of LaTeX
% version 2005/12/01 or later.
%
% This work has the LPPL maintenance status `maintained'.
%
% The Current Maintainer of this work is Tom M. Ragonneau.
\documentclass{polyu-presentation}
\usepackage{microtype}

% List of hyphenation exceptions for US English
% Source: https://ctan.org/tex-archive/info/digests/tugboat/hyphenex
\input{ushyphex}

% Bibliographical resources
\addbibresource{ragonneau-bib/strings.bib}
\addbibresource{ragonneau-bib/optim.bib}

% Dedicated mathematical macros
\newcommand{\auglag}{\mathcal{L}_{\mathsf{A}}}
\newcommand{\auglagalt}{\widetilde{\mathcal{L}}_{\mathsf{A}}}
\newcommand{\con}[1]{c_{#1}}
\newcommand{\conm}[2][]{\hat{c}_{#2}\ifthenelse{\equal{#1}{}}{}{^{#1}}}
\newcommand{\fset}{\Omega}
\newcommand{\ieq}{\mathcal{E}}
\newcommand{\iub}{\mathcal{I}}
\newcommand{\iter}[1][]{x\ifthenelse{\equal{#1}{}}{}{^{#1}}}
\newcommand{\lag}[1][]{\mathcal{L}\ifthenelse{\equal{#1}{}}{}{^{#1}}}
\newcommand{\lagalt}[1][]{\widetilde{\mathcal{L}}\ifthenelse{\equal{#1}{}}{}{^{#1}}}
\newcommand{\lagm}[1][]{\widehat{\mathcal{L}}\ifthenelse{\equal{#1}{}}{}{^{#1}}}
\newcommand{\lagp}[1][]{L\ifthenelse{\equal{#1}{}}{}{_{#1}}}
\newcommand{\lm}[1][]{\lambda\ifthenelse{\equal{#1}{}}{}{^{#1}}}
\newcommand{\merit}[1][]{\varphi\ifthenelse{\equal{#1}{}}{}{^{#1}}}
\newcommand{\meritm}[1][]{\widehat{\varphi}\ifthenelse{\equal{#1}{}}{}{^{@@#1}}}
\newcommand{\nstep}[1][]{n\ifthenelse{\equal{#1}{}}{}{^{#1}}}
\newcommand{\nstepalt}[1][]{\bar{n}\ifthenelse{\equal{#1}{}}{}{^{#1}}}
\newcommand{\obj}{f}
\newcommand{\objm}[1][]{\hat{f}\ifthenelse{\equal{#1}{}}{}{^{#1}}}
\newcommand{\objmalt}[1][]{\tilde{f}\ifthenelse{\equal{#1}{}}{}{^{#1}}}
\newcommand{\pstep}[1][]{p\ifthenelse{\equal{#1}{}}{}{^{#1}}}
\newcommand{\rad}[1][]{\Delta\ifthenelse{\equal{#1}{}}{}{@\!^{#1}}}
\newcommand{\radlb}[1][]{\delta\ifthenelse{\equal{#1}{}}{}{^{#1}}}
\newcommand{\ratio}[1][]{\rho\ifthenelse{\equal{#1}{}}{}{^{#1}}}
\newcommand{\rstep}[1][]{r\ifthenelse{\equal{#1}{}}{}{^{#1}}}
\newcommand{\sstep}[1][]{s\ifthenelse{\equal{#1}{}}{}{^{#1}}}
\newcommand{\step}[1][]{d\ifthenelse{\equal{#1}{}}{}{^{#1}}}
\newcommand{\tstep}[1][]{t\ifthenelse{\equal{#1}{}}{}{^{#1}}}
\newcommand{\xl}{l}
\newcommand{\xpb}[1][]{\mathcal{P}}
\newcommand{\xpt}[1][]{\mathcal{Y}\ifthenelse{\equal{#1}{}}{}{^{#1}}}
\newcommand{\xsv}[1][]{\mathcal{S}}
\newcommand{\xu}{u}

% Performance and data profiles
\usepackage{xstring}
\newcommand{\drawprofiles}[4]{%
    \def\selectsolvers{#2}%
    \def\selectcsv{figures/#3}%
    \def\selectprofile{#4}%
    \ifthenelse{\equal{#1}{performance}}{%
        \def\selectxlabel{$\log_2(\alpha)$}%
        \def\selectylabel{Performance profiles~$\rho_s(\alpha)$}%
    }{%
        \def\selectxlabel{Number of simplex gradients~$\alpha$}%
        \def\selectylabel{Data profiles~$d_s(\alpha)$}%
    }
    \input{figures/profiles.tex}%
}
\newcommand{\drawperformanceprofiles}[3]{\drawprofiles{performance}{#1}{#2}{#3}}
\newcommand{\drawdataprofiles}[3]{\drawprofiles{data}{#1}{#2}{#3}}

\title{Model-Based DFO Methods and Software}
\subtitle{Ph.D. thesis defense}
\author[Tom M. Ragonneau]{\texorpdfstring{
    Tom M. Ragonneau\\
    \footnotesize Co-supervised by Dr.\ Zaikun Zhang and Prof.\ Xiaojun Chen
}{Tom M. Ragonneau}}
\institute[PolyU AMA]{
    Department of Applied Mathematics\\
    The Hong Kong Polytechnic University
}
\date{December 6, 2022}
\titlegraphic{\href{https://www.tomragonneau.com/}{\includegraphics[width=0.8in]{images/qr/personal.png}}}

\begin{document}

\begin{frame}
	\titlepage
\end{frame}

\begin{frame}
    \frametitle{Table of contents}
    
	\tableofcontents[hideallsubsections]
\end{frame}

\begin{frame}
    \frametitle{A Byrd-Omojokun approach}

    To define the \alert{trial step}~$\step[k]$, we decompose~$\step[k] = \nstep[k] + \tstep[k]$, where
    \begin{enumerate}
        \item the \alert{normal step}~$\nstep[k]$ reduces the (possible) constraint violation, and
        \item the \alert{tangential step}~$\tstep[k]$ reduces the objective function of the subproblem.
    \end{enumerate}
    
    \begin{columns}
        \begin{column}{0.45\textwidth}
            \begin{center}
                \begin{tikzpicture}
                    % Linear constraints
                    \uncover<1-6>{\fill[color=RoyalBlue,opacity=0.4] (-4,-1) -- (-1.5,-1) -- (-0.5,4) -- (-4,4) -- cycle;}
                    \uncover<7>{\fill[color=RoyalBlue,opacity=0.4] (-4,-1) -- (-2.1,-1) -- (-1.1,4) -- (-4,4) -- cycle;}
                    \uncover<1,2>{\fill[color=RoyalBlue,opacity=0.4] (-4,1) -- (0,4) -- (-4,4) -- cycle;}
                    \uncover<3->{\fill[color=RoyalBlue,opacity=0.4] (-4,0.125) -- (1,3.875) -- (1,4) -- (-4,4) -- cycle;}
        
                    % Trust regions
                    \begin{scope}
                        \clip (-4,-1) rectangle (1,4);
                        \draw[fill=Dandelion,draw opacity=0.7,fill opacity=0.5] (0,0) circle (3);
                        \draw[densely dotted,fill=Dandelion,opacity=0.7] (0,0) circle (2.5);
                    \end{scope}
        
                    % Feasible region for the tangential subproblem
                    \begin{scope}
                        \clip (-4,0.125) -- (-27/34,43/17) -- (-0.5,4) -- (-4,4) -- cycle;
                        \uncover<4-6>{\fill[pattern=north west lines,opacity=0.7] (0,0) circle (3);}
                    \end{scope}
                    \begin{scope}
                        \clip (-4,0.125) -- (-1.5,2) -- (-1.1,4) -- (-4,4) -- cycle;
                        \uncover<7>{\fill[pattern=north west lines,opacity=0.7] (0,0) circle (3);}
                    \end{scope}
        
                    % Frame and annotations
                    \uncover<5>{
                        \draw[-stealth,thick,OliveGreen] (-1.5,2) -- (-0.9,2.7);
                        \draw[-stealth,thick] (0,0) -- (-0.9,2.7);
                        \node[below,xshift=5pt,text=OliveGreen] at (-1.2,2.35) {$\tstep[k]$};
                        \node[above right] at (-0.45,1.35) {$\step[k]$};
                    }
                    \uncover<2->{
                        \draw[-stealth,thick,Mahogany] (0,0) -- (-1.5,2);
                        \node[below left,text=Mahogany] at (-0.75,1) {$\nstep[k]$};
                    }
                    \draw[fill] (0,0) circle (1.4pt) node[below right] {$\iter[k]$};
                    \draw[thick] (-4,-1) rectangle (1,4);
                \end{tikzpicture}
            \end{center}
        \end{column}
        \begin{column}{0.55\textwidth}
            \begin{tikzpicture}
                \draw[fill=Dandelion,draw opacity=0.7,fill opacity=0.5] (0,0) circle (.2);
                \node[right] at (.4,0) {Trust region};
                \draw[densely dotted,fill=Dandelion,opacity=0.7] (0,-0.6) circle (.2);
                \node[right] at (.4,-0.6) {Reduced trust region};
                \fill[color=RoyalBlue,opacity=0.4] (-.2,-1.4) rectangle (.2,-1);
                \node[right] at (.4,-1.2) {Linear constraints};
                \uncover<4->{
                    \fill[pattern=north west lines,opacity=0.7] (-.2,-2) rectangle (.2,-1.6);
                    \node[right] at (.4,-1.8) {Feasible region for~$\tstep[k]$};
                }
            \end{tikzpicture}

            \bigskip

            Our \alert<1-6>{new} approach vs. the \alert<7>{standard}\footfullcite{Conn_Gould_Toint_2000} one.
        \end{column}
    \end{columns}
\end{frame}

\appendix

\begin{frame}[t,allowframebreaks]
    \frametitle{References}

	\printbibliography
\end{frame}

\end{document}